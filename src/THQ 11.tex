\documentclass{article}
\usepackage[utf8]{inputenc}
\usepackage[margin=1in]{geometry}
\usepackage{amsmath}
\usepackage{amssymb}
\usepackage{enumitem}
\usepackage{amsfonts}


\title{Take Home Quiz 11}
\author{}
\date{}

\newcommand{\dint}[1]{\iint_{\mathcal{D}} #1 dA}
\newcommand{\D}{\mathcal{D}}


\begin{document}
    \maketitle

    \paragraph{1} Let $S$ be the surface of the cone $z=-\sqrt{x^2+y^2}$ above the plane $z=-2$.
    \begin{enumerate}[label=(\alph*)]
        \item Write a parametrization of this surface. Be sure to include the domain as part of your parametrization. Restrict the domain so that we get exactly one copy of $S$. (Warning: read part (b) before you do this so that you choose a parametrization that will work for part b.)
        \item Use your parametrization to calculate the surface area of $S$.
    \end{enumerate}
    \begin{itemize}
        \item \textbf{Solution}: It is tempting to use $\mathbf{r}(u,v) = \langle u,v, \sqrt{u^2+v^2} \rangle$ but note that the partials will be discontinuous on the domain with which we are working ($\mathcal{D}$ being the disk of radius 2). So we will use polar instead.
        \item Let $\mathbf{r}(r, \theta) = \langle r\cos(\theta), r\sin(\theta), -r \rangle$ for $r \in [0,2]$ and $\theta \in [0,2\pi]$. The partials are
        \[ \mathbf{r}_r = \langle \cos(\theta), \sin(\theta), -1 \rangle \hspace{0.3in} \mathbf{r}_{\theta} = 
        \langle -r\sin(\theta), r\cos(\theta), 0 \rangle\]
        Computing the magnitude of the normal, we get
        \begin{align*}
            |\mathbf{n}| &= |\mathbf{r}_r \times \mathbf{r}_{\theta}| \\
            &= \begin{vmatrix} 
                \mathbf{i} & \mathbf{j} & \mathbf{k} \\
                \cos(\theta) & \sin(\theta) & -1 \\
                -r\sin(\theta) & r\cos(\theta) & 0 \\
            \end{vmatrix} \\
            &= |\langle r\cos(\theta), r\sin(\theta) ,-r\sin^2(\theta)-r\cos^2(\theta) \rangle| \\
            &= \sqrt{r^2 + r^4} \\
            &= r\sqrt{1+r^2}
        \end{align*}
        \item We know that $r$ is always positive over the interval $[0,2]$ so we can pull a positive $r$ out of the square root. The integral for surface area becomes
        \begin{align*}
            \iint_{\mathcal{D}} |\mathbf{r}_r \times \mathbf{r}_{\theta}| dA &= \int_0^{2\pi} \int_0^2 \sqrt{1+r^2}rdrd\theta \\
            &w = 1+r^2 \hspace{0.1in} dw = 2rdr\\
            &w(0) = 1 \hspace{0.1in} w(2) = 5\\
            &= \frac{1}{2}\int_0^{2\pi} \int_1^5 \sqrt{w}dw \\
            &= \frac{1}{2}(2\pi)[\frac{2}{3}w^{3/2}]_1^5 \\
            &= \frac{2\pi}{3}(5^{3/2}-1)
        \end{align*}
    \end{itemize}
    
    \newpage
    \paragraph{2} Let $S$ be the unit sphere.
    \begin{enumerate}[label=(\alph*)]
        \item Write a parameterization for $S$ in terms of the parameters $\theta$ and $\phi$, which take on their typical spherical coordinate roles in $\mathbb{R}^3$. Be sure to restrict the domain appropriately such that we get exactly one copy of $S$.
        \item Use your parameterization from part (a) to find the equation of a tangent line at the point
        \[ (\frac{\sqrt{3}}{2}, \frac{1}{2}, 0) \]
    \end{enumerate}
    \begin{itemize}
        \item \textbf{Solution}: We will use the standard rectangular to spherical coordinate conversion system to build the paramterization, with $\rho=1$:
        \[ \mathbf{r}(\theta, \phi) = \langle \sin(\phi)\cos(\theta), \sin(\phi)\sin(\theta), \cos(\phi) \rangle \]
        With the restrictions $0 \leq \theta \leq 2\pi$ and $0 \leq \phi \leq \pi$. 
        \item The partials are
        \[ \mathbf{r}_{\theta} = \langle -\sin(\phi)\sin(\theta), \sin(\phi)\cos(\theta), 0 \rangle \hspace{0.3in} \mathbf{r}_{\phi} = \langle \cos(\phi)\cos(\theta), \cos(\phi)\sin(\theta), -\sin(\phi) \rangle\]
        Computing the normal, we get
        \[
            \mathbf{n} = \mathbf{r}_{\theta} \times \mathbf{r}_{\phi} 
            = \begin{bmatrix} 
                \mathbf{i} & \mathbf{j} & \mathbf{k} \\
                -\sin(\phi)\sin(\theta) & \sin(\phi)\cos(\theta) & 0 \\
                \cos(\phi)\cos(\theta) & \cos(\phi)\sin(\theta) & -\sin(\phi) \\
            \end{bmatrix} =
            \begin{bmatrix}
                -\sin^2(\phi)\cos(\theta)\\
                -\sin^2(\phi)\sin(\theta)\\
                -\sin(\phi)\cos(\phi)
            \end{bmatrix}
        \]
        \item The point of tangency in $\mathbb{R}^3$ corresponds to some point in the $\theta\phi$ plane, governed by the following equations:
        \begin{align*}
            \sin(\phi_0)\cos(\theta_0) &= \frac{\sqrt{3}}{2}\\
            \sin(\phi_0)\sin(\theta_0) &= \frac{1}{2}\\
            \cos(\phi_0) &= 0\\
        \end{align*}
        $\cos(\phi_0) = 0$ implies $\phi_0 = \pi/2$ or $3\pi/2$, but we only care about $\pi/2$ for our interval. By the first equation, $\sin(\pi/2)\cos(\theta_0) = \cos(\theta_0) = \sqrt{3}/2$ means $\theta_0 = \pi/6$ or $11\pi/6$. Since by the second equation $\sin(\theta_0) > 0$, we know $\theta_0$ must be in quadrant I. The point is $(\theta_0, \phi_0)=(\pi/6, \pi/2)$.
        \item To find the normal of our tangent plane, we need to evaluate $\mathbf{n}(\pi/6, \pi/2)$:
        \[ 
        \mathbf{n}=
        \begin{bmatrix}
            -\sin^2(\pi/2)\cos(\pi/6)\\
            -\sin^2(\pi/2)\sin(\pi/6)\\
            -\sin(\pi/2)\cos(\pi/2)
        \end{bmatrix}=
        -\begin{bmatrix}
            \sqrt{3}/2\\1/2\\0
        \end{bmatrix}
        \]
        \item Putting everything in point normal form we get
        \[ \frac{\sqrt{3}}{2}(x-\frac{\sqrt{3}}{2}) + \frac{1}{2}(y-\frac{1}{2}) = 0 \]
    \end{itemize}

\end{document}


