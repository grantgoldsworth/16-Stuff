\documentclass{article}
\usepackage[utf8]{inputenc}
\usepackage[margin=1in]{geometry}
\usepackage{amsmath}
\usepackage{amssymb}
\usepackage{enumitem}
\usepackage{hyperref}
\usepackage{amsfonts}

\title{Section 16.3 Homework}
\author{1-25 odd, 29--35 odd}
\date{}

\newcommand{\eint}[1]{\iiint_E #1 dV}
\newcommand{\parf}[2]{\frac{\partial #1}{\partial #2}}
\newcommand{\D}{\mathcal{D}}


\begin{document}
    \maketitle

    \paragraph{1} Since the gradient of $f$ is continuous, we know that the work line integral of the gradient is just the difference in $z$ values: $50-10=40$.

    \section*{3-9} Determine whether or not $F$ is a conservative vector field.
    If it is, find a function $f$ such that $\mathbf{F} = \nabla f$.

    \paragraph{3}
    \[  \mathbf{F}(x,y) = <(xy+y^2), (x^2+2xy)> \]
    \textbf{Solution}: From our notes, if the mixed second partials of the potential function candidate are equal
    (basically if the partials of the component functions of $\mathbf{F}$ are equal), then the vector field is conservative, and
    we can go about finding the potential function.
    Let $P(x,y) = xy+y^2$ and $Q(x,y) = x^2+2xy$.
    We have that
    \begin{align*}
        \parf{P}{y} &= x + 2y \\
        \parf{Q}{x} &= 2x + 2y
    \end{align*}
    The partials are not equal and thus $\mathbf{F}$ is not a conservative vector field.


    \paragraph{5}
    \[\mathbf{F(x,y)}=<y^{2}e^{xy}, (1+xy)e^{xy}>\]
    \textbf{Solution}: Same as above, take partials of component functions of the vector field
    \begin{align*}
        \parf{P}{y} &= 2ye^{xy} + y^{2}xe^{xy} \\
        \parf{Q}{x} &= ye^{xy} + ye^{xy} + xy^{2}e^{xy}\\
    \end{align*}
    These partials are equal on their natural domains which is all of $\mathbb{R}^2$, a simply connected open region.
    Therefore, a potential function exists.
    To find this function, we will start by integrating $P(x,y)$ against $x$:
    \[f(x,y)=\int P(x,y)dx = \int (y^{2}e^{xy}) dx = ye^{xy} + g(y)\]
    Take the derivative of $f$ with respect to $y$ and compare to $Q(x,y)$:
    \begin{align*}
        \parf{f}{y} &= Q(x,y) \\
        e^{xy} + xye^{xy} + g'(y) &= e^{xy} + xye^{xy} \\
        g'(y) &= 0 \rightarrow g(y) = 0 + C \\
    \end{align*}
    The general solution is $f(x,y) = ye^{xy} + C$.
    
    \paragraph{7}

    \paragraph{7}
    \[\mathbf{F}=<ye^x + \sin(y), e^x + x\cos(y)\]
    \textbf{Solution}: Test if $\mathbf{F}$ is conservative:
    \[\parf{P}{y} = e^x + \cos(y) \hspace{0.3in} \parf{Q}{x} = e^x + \cos(y)\]
    The domain of $\mathbf{F}$ is all of $\mathbb{R}^2$.
    The vector field is conservative. Find a potential function:
    \begin{align*}
        f &= \int (ye^{x} + \sin(y))dx \\
        &= ye^{x} + x\sin(y) + g(y) \\
        f_y &= Q(x,y)\\
        e^{x} + x\cos(y) +g'(y) &= e^x + x\cos(y) \\
        g'(y) &= 0 \rightarrow g(y) = C \\
    \end{align*}
    The general solution is $f(x,y) = ye^{x} + x\sin(y) + C$

    \paragraph{9}
    \[\mathbf{F(x,y)}=<y^{2}\cos(x) + \cos(y), 2y\sin(x)-x\sin(y)>\]
    \textbf{Solution}: $\mathbf{F}$ is continuous on all of $\mathbb{R}^2$. Test if it is conservative:
    \[ \parf{P}{y} = 2y\cos(x)-\sin(y) \hspace{0.3in} \parf{Q}{x} = 2y\cos(x)-\sin(y)\]
    Find a potential function:
    \begin{align*}
        f &= \int P(x,y) dx \\
        &= y^{2}\sin(x) +x\cos(y) + g(y) \\
        f_y &= Q(x,y)\\
        2y\sin(x) -x\sin(y) + g'(y) &= 2y\sin(x) - x\sin(y) \\
        g'(y) &= 0 \rightarrow g(y) = C
    \end{align*}
    The general solution is $f(x,y) = y^{2}\sin(x) + x\cos(x) + C$

    \paragraph{11} \textbf{Solution}: $\mathbf{F}=<2xy, x^2>$ has domain of all of the $xy$ plane, and the first partials of the
    component functions are equal ($P_y = Q_x = 2x$).
    Therefore, the vector field is conservative and line integrals of conservative vector field are independent of path. To find the value of all three
    line work integrals, we can find a potential function $f$:
    \begin{align*}
        f &= \int P(x,y)dx \\
        &= x^{2}y + g(y)\\
        f_y &= Q(x,y) \\
        x^2 + g'(y) &= x^2 \rightarrow g'(y) = 0 \\
        g(y) &= C
    \end{align*}
    Choose $C=0$, and the value of all three curves is $f(3,2)-f(1,2)=16$.

    \section*{13-17} Find a function $f$ such that $\mathbf{F} = \nabla f$, and use this function to evaluate a line
    work integral along the given path.

    \paragraph{13}
    \[\mathbf{F(x,y)}=<x^{2}y^{3}, x^{3}y^{2}> \hspace{0.3in} C: \mathbf{r}(t) = <t^3-2t, t^3+2t> (t\in [0,1])\]
    \textbf{Solution}: I'll assume that because of the way these problems are given, $\mathbf{F}$ is a conservative vector field and
    we can jump straight to finding $f$:
    \begin{align*}
        f &= \int P(x,y) dx \\
        &= \int x^{2}y^{3}dx = \frac{1}{3}x^{3}y^{3} + g(y) \\
        f_y &= Q(x,y)\\
        x^{3}y^{2} + g'(y) &= x^{3}y^{2} \rightarrow g'(y) = 0\\
        f(x,y) &= \frac{1}{3}x^{3}y^{3} + C
    \end{align*}
    The path travels from (0,0) $(t=0)$ to $(-1,3)$ $(t=1)$.
    \[\int_C \mathbf{F\cdot}d\mathbf{r} = f(-1,3)-f(0,0) = -9\]

    \paragraph{15}
    \[
        \mathbf{F(x,y,z)} = \begin{bmatrix} yz \\ xz \\ xy+2z \end{bmatrix}
    \]
    Where $C$ is the line segment from $(1,0,-2)$ to $(4,6,3)$.\\
    \textbf{Solution}: We could parameterize $C$ easily but we don't need to if $\mathbf{F}$ is conservative. As with
    the last problem, I'll assume it is from the way it is presented in this problem set. Let's find the general potential function:
    \begin{align*}
        f &= \int P(x,y,z)dx = xyz + g(y,z) \\
        f_y &= Q(x,y) \\
        xz + g_y(y,z) &= xz \rightarrow g_y(y,z) = 0 \\
        g(y,z) &= h(z)\\
        f_z(x,y,z) &= R(x,y,z)\\
        xy + h'(z) &= xy + 2z \rightarrow h'(z) = 2z \\
        h(z) &= z^2 + C\\
    \end{align*}
    Our general solution is $f(x,y,z) = xyz + z^2 + C$. Choosing $C=0$, the result of any line integral along a path connecting the
    two points is $f(4,6,3)-f(1,0,-2)=81-4=77$'

    \paragraph{17}
    \[
        \mathbf{F(x,y,z)} = \begin{bmatrix} yze^{xz} \\ e^{xz} \\ xye^{xz} \end{bmatrix}
        \hspace{0.3in}
        C: \mathbf{r}(t) = \begin{bmatrix} t^2 + 1\\ t^2-1 \\ t^2-2t \end{bmatrix} (t \in [0,2])
    \]
    Can't scare me with that path parameterization.
    We will use it to calculate our start and end points, however: $(1,-1,0)$ to $(5,3,0)$.
    \begin{align*}
        f &= \int P(x,y,z) dx \\
        f &= ye^{xz} + g(y,z)\\
        f_y &= Q(x,y,z) \\
        e^{xz} + g_y(y,z) &= e^{xz} \rightarrow g_y(y,z) = 0 \\
        g(y,z) &= h(z) \\
        f_z &= R(x,y,z) \\
        xye^{xz} + h'(z) &= xye^{xz} \rightarrow h'(z) = 0 \\
        h(z) = C \\
    \end{align*}
    The general solution is $f(x,y,z) = ye^{xz}+C$. Choosing $C=0$, the result of a line integral along any path connecting
    the two points of the curve is $f(5,3,0)-f(1,-1,0)=3e^0 +e^0 = 4$
    
    \paragraph{19} Show that the line integral is independent of path and evaluate it.
    \[\int_C 2xe^{-y}dx + (2y-x^{2}e^{-y})dy\]
    Where $C$ is any path from (1,0) to (2,1).\\
    \textbf{Solution}: The integrand is a physics-abomination mess, but it is the result of a dot product. More specifically,
    $\mathbf{F} \cdot d\mathbf{r}$ with $\mathbf{F}=<2xe^{-y}, 2y-x^{2}e^{-y}>$ and $d\mathbf{r} = <dx, dy>$. If we can show that the
    vector field is conservative, we can find a potential function and answer the question.
    \[P_y = -2xe^{-y} \hspace{0.3in} Q_x = -2xe^{-y}\]
    Hooray, they are equal. Let's find the potential function:
    \begin{align*}
        f &= \int P(x,y) dx = x^{2}e^{-y} + g(y) \\
        f_y &= Q(x,y) \\
        -x^{2}e^{-y} + g'(y) &= 2y-x^{2}e^{-y} \rightarrow g'(y) = 2y \\
        g(y) &= \int 2y dy = y^2 + C \\
    \end{align*}
    The general solution is $f(x,y) = x^{2}e^{-y} + y^2 + C$.
    The work line integral along any path will equal $f(2,1)-f(1,0)=(4/e+1)-(1)=4/e$

    \paragraph{21} Suppose you're asked to determine the curve that requires
    the least work for a force field $\mathbf{F}$ to move a particle from one point to another point.
    You decide to check first whether F is conservative, and indeed it turns out that it is.
    How would you reply to the request?\\
    \textbf{Solution}: I'd say "it's up to you chief, it don't matter"

    \paragraph{23} Find the work done by the force field $\mathbf{F}$ in moving an object from $P$ to $Q$
    \[\mathbf{F}(x,y) = <x^3, y^3> \hspace{0.3in}P(1,0),Q(2,2)\]
    \textbf{Solution} Is the vector field conservative? Most likely not, but let's give it a shot:
    \[P_y = 0 \hspace{0.3in} Q_x = 0\]
    The vector field is continuous on a simply connected open region (all of $\mathbb{R}^2$) and the first partials are equal.
    I stand corrected.
    \begin{align*}
        f &= \int P(x,y)dx = \frac{1}{4}x^4 + g(y)\\
        g'(y) &= y^3 \\
        g(y) &= \frac{1}{4}y^4 + C\\
    \end{align*}
    $f(x,y) = x^4/4 + y^4/4 + C$. Choose $C=0$ and the final answer is $f(2,2)-f(1,0)=8-1/4=31/4$

    \paragraph{25} \textbf{Solution}: If we take a curve $C$ to be a circle oriented CCW (along the vectors), then the net
    result of a line integral along that path would not be 0 -- the field is not conservative (the vectors are somewhat parallel with the
    CCW orientation, and the dot product $\mathbf{F \cdot T}$ would be returning non-zero numbers).
    
    \paragraph{29} Show that if the vector field $\mathbf{F}(x,y,z)=<P,Q,R>$ is conservative, and $P,Q,R$ have continuous first order partial
    derivatives, then
    \[
        \parf{P}{y} = \parf{Q}{x} \hspace{0.3in}
        \parf{P}{z} = \parf{R}{x} \hspace{0.3in}
        \parf{Q}{z} = \parf{R}{y}
    \]
    \textbf{Solution}: This is basically asking to show that $\nabla \times \mathbf{F}$ is $\mathbf{0}$, I think (0 curl
    is a property of conservative vector fields).
    \[
        \nabla \times \mathbf{F} =
        \begin{vmatrix}
            \mathbf{i} & \mathbf{j} & \mathbf{k}\\
            \parf{}{x} & \parf{}{y} & \parf{}{z} \\
            P & Q & R
        \end{vmatrix} =
        \begin{bmatrix}
            R_y - Q_z \\ P_z - R_x \\ Q_x - P_y \\
        \end{bmatrix}
        = \mathbf{0}
    \]
    In order for the last equal sign to be true (for the determinant vector to equal the zero vector) the equalities presented
    in the problem need to be true (so the differences are 0).
    This is the fancy way of saying that by Clairaut's theorem, the continuous second mixed partials on an open connected region
    are equal (we know this applies because $\mathbf{F}$ is conservative and $\mathbf{F} = \nabla f$).


    \section*{31,33} Determine whether the set is (a) open, (b) connected, (c) simply connected
    \paragraph{31}
    \[ \{(x,y)|\hspace{0.1in} 0 < y < 3\}\]
    \textbf{Solution}:
    \begin{enumerate}[label=(\alph*)]
        \item The set has non-inclusive boundaries (indicated by the less than symbols), so it is open.
        \item The set is connected because it is "in one piece".
        \item The set is simply connected because there are "no holes"; no simple closed curve in the set encloses points not in the set.
    \end{enumerate}

    \paragraph{33}
    \[ \{(x,y)|\hspace{0.1in} 1 \leq x^2 + y^2 \leq 4, y \geq 0\}\]
    \textbf{Solution}:
    \begin{enumerate}[label=(\alph*)]
        \item The set has inclusive boundaries (I can place a disk at a point in the set that goes outside the set) and thus it is not open.
        \item The set is connected because it is in one piece.
        \item The set is simply connected because no simple closed curve in the set encloses points in the set
    \end{enumerate}

    \paragraph{35}
    \[\mathbf{F}(x,y) = \frac{1}{x^2 + y^2}<-y, x>\]
    \begin{enumerate}[label=(\alph*)]
        \item Show $P_y = Q_x$
        \item Show that a work line integral of $\mathbf{F}$ is not independent of path.
    \end{enumerate}
    \textbf{Solution}:
    \begin{align*}
        P_y &= \frac{-(x^2+y^2)-2y^2)}{(x^2+y^2)^2} = \frac{y^2-x^2}{x^2+y^2}\\
        Q_x &= \frac{-(x^2+y^2)-2x^2}{(x^2+y^2)^2} = \frac{y^2-x^2}{x^2+y^2}
    \end{align*}
    The partials are equal, but the problem asks us to show that $\mathbf{F}$ is not independent of path.
    The book suggests using two halves of
    the unit circle as paths, and showing that the line integrals are equal.
    Let $C_1: \mathbf{r}(t) = <\cos(t), \sin(t)>$ for $t \in [0,\pi]$
    and $C_2: \mathbf{r}(t) = <\cos(t), \sin(t)>$ for $t: 2\pi \to \pi$.
    \begin{align*}
        \int_{C_1} \mathbf{F} \cdot d \mathbf{r} &= \int_{0}^{\pi} \mathbf{F(r}(t)) \cdot <-\sin(t), \cos(t)>dt\\
        &= \int_{0}^{\pi} <-\sin(t), \cos(t)> \cdot <-\sin(t), \cos(t)>dt \\
        &= \int_0^{\pi}1dt = \pi
    \end{align*}
    The work line integral over $C_2$ will have the same calculation but with different bounds:
    \[\int_{C_2} \mathbf{F(r}(t)) \cdot d\mathbf{r} = \int_{2\pi}^{\pi} 1 dt = \pi - 2\pi = -\pi\]
    The line integrals are not equal and therefore are not independent of path.
    Since we did not integrate over a path enclosing the hole at the origin, we are not in violation of our rules.
    \paragraph{35, Note}: The textbook mentions using a contrapositive of Theorem 3. Theorem 3 states that $\int_C \mathbf{F} \cdot d\mathbf{r}$ is independent
    of path if and only if $\int_C \mathbf{F} \cdot d\mathbf{r} = 0$ for every closed path $C$ in $D$.
    The contrapositive would be that any line integral over a closed path in $C$ of a vector function that isn't independent of path is not 0. For example,
    if we integrate over the entire unit circle, the result is $2\pi \neq 0$.
\end{document}