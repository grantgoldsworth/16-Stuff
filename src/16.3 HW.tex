\documentclass{article}
\usepackage[utf8]{inputenc}
\usepackage[margin=1in]{geometry}
\usepackage{amsmath}
\usepackage{amssymb}
\usepackage{enumitem}
\usepackage{hyperref}
\usepackage{amsfonts}

\title{Section 16.3 Homework}
\author{1-25 odd, 29--35 odd}
\date{}

\newcommand{\eint}[1]{\iiint_E #1 dV}
\newcommand{\parf}[2]{\frac{\partial #1}{\partial #2}}
\newcommand{\D}{\mathcal{D}}


\begin{document}
    \maketitle

    \paragraph{1} Since the gradient of $f$ is continuous, we know that the work line integral of the gradient is just the difference in $z$ values: $50-10=40$.

    \section*{3-9} Determine whether or not $F$ is a conservative vector field.
    If it is, find a function $f$ such that $\mathbf{F} = \nabla f$.

    \paragraph{3}
    \[  \mathbf{F}(x,y) = <(xy+y^2), (x^2+2xy)> \]
    \textbf{Solution}: From our notes, if the mixed second partials of the potential function candidate are equal
    (basically if the partials of the component functions of $\mathbf{F}$ are equal), then the vector field is conservative, and
    we can go about finding the potential function.
    Let $P(x,y) = xy+y^2$ and $Q(x,y) = x^2+2xy$.
    We have that
    \begin{align*}
        \parf{P}{y} &= x + 2y \\
        \parf{Q}{x} &= 2x + 2y
    \end{align*}
    The partials are not equal and thus $\mathbf{F}$ is not a conservative vector field.


    \paragraph{5}
    \[\mathbf{F(x,y)}=<y^{2}e^{xy}, (1+xy)e^{xy}>\]
    \textbf{Solution}: Same as above, take partials of component functions of the vector field
    \begin{align*}
        \parf{P}{y} &= 2ye^{xy} + y^{2}xe^{xy} \\
        \parf{Q}{x} &= ye^{xy} + ye^{xy} + xy^{2}e^{xy}\\
    \end{align*}
    These partials are equal on their natural domains which is all of $\mathbb{R}^2$, a simply connected open region.
    Therefore a potential function exists.
    To find this function, we will start by integrating $P(x,y)$ against $x$:
    \[f(x,y)=\int P(x,y)dx = \int (y^{2}e^{xy}) dx = ye^{xy} + g(y)\]
    Take the derivative of $f$ with respect to $y$ and compare to $Q(x,y)$:
    \begin{align*}
        \parf{f}{y} &= Q(x,y) \\
        e^{xy} + xye^{xy} + g'(y) &= e^{xy} + xye^{xy} \\
        g'(y) &= 0 \rightarrow g(y) = 0 + C \\
    \end{align*}
    The general solution is $f(x,y) = ye^{xy} + C$.

\end{document}