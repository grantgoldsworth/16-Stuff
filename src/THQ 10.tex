\documentclass{article}
\usepackage[utf8]{inputenc}
\usepackage[margin=1in]{geometry}
\usepackage{amsmath}
\usepackage{amssymb}
\usepackage{enumitem}
\usepackage{amsfonts}


\title{Take Home Quiz 10}
\author{}
\date{}

\newcommand{\dint}[1]{\iint_{\mathcal{D}} #1 dA}
\newcommand{\D}{\mathcal{D}}


\begin{document}
    \maketitle

    \paragraph{1} Let $C$ be the top half of the unit circle, oriented \textit{clockwise}.
    Let $\mathbf{F}(x,y)=\langle y,x^2\rangle$ be a force field.
    Calculate the work done by $\mathbf{F}$ acting on a particle that moves along $C$.
    Assume force is measured in Newtons and distance in meters.
    \\
    \begin{itemize}
        \item \textbf{Solution}: $C$ can be parameterized by $\mathbf{r}(t)=\langle \cos(t), \sin(t)\rangle $ for $t \in [0,\pi]$.
        Since $\mathbf{F}$ is not conservative, we have to calculate the integral manually.
        $\mathbf{F(r}(t)) = \langle \sin(t), \cos^2(t)\rangle $ and $d\mathbf{r}(t)=\langle -\sin(t), \cos(t)\rangle dt$.
        \begin{align*}
            \int_C \mathbf{F\cdot }d\mathbf{r} &= \int_0^{2\pi} \mathbf{F(r}(t)) \cdot \mathbf{r'(t)}dt \\
            &= \int_0^{2\pi} \langle \sin(t), \cos^{2}(t)\rangle  \cdot \langle -\sin(t), \cos(t)\rangle dt \\
            &= \int_0^{2\pi} (-\sin^{2}(t) + \cos^{3}(t))dt \\
            &= \int_0^{2\pi} \cos^{3}(t)dt - \int_0^{2\pi} \sin^{2}(t)dt \\
        \end{align*}
        \item For the first integral, we split the cubic power of cosine to get $\cos^{2}(t)=1-\sin^{2}(t)$.
        For the second integral, we can derive a double angle relation for $\sin^{2}(t)$:
        \begin{align*}
            e^{2ti} &= (\cos(t) + i\sin(t))(\cos(t) + i\sin(t)) \\
            &= \cos^{2}(t) - \sin^2(t) + 2i\sin(t)\cos(t) \\
            \cos(2t) &= \mathrm{Re}(e^{2ti}) = \cos^2(t)-\sin^2(t)\\
            \cos(2t) &= (1-\sin^{2}(t))-\sin^{2}(t) = 1-2\sin^{2}(t)\\
            \sin^2(t) &= \frac{1-\cos(2t)}{2}
        \end{align*}
        Back to our integrals:
        \begin{align*}
            \int_0^{2\pi} \cos^{3}(t)dt - \int_0^{2\pi} \sin^{2}(t)dt &= \int_0^{2\pi} (1-\sin^{2}(t))\cos(t)dt - \int_0^{2\pi}\frac{1-\cos(2t)}{2}dt \\
            &u = \sin(t) \hspace{0.2in} du = \cos(t)dt \\
            &u(0) = 0 \hspace{0.1in} u(2\pi) = 0\\
            &= 0 - \frac{1}{2}[t-\frac{1}{2}\sin(2t)]_0^{2\pi} = \pi \hspace{0.1in} \mathrm{N\cdot m}
        \end{align*}
    \end{itemize}

    \newpage

    \paragraph{2}
    \[\mathbf{F} = \left\langle \frac{2x}{y}-e^{x}, -\frac{x^2}{y^2}+3y\right\rangle \]
    \begin{itemize}
        \item Draw the largest simply connected region $D$ that contains the point (1,5) on which $\mathbf{F}$ is defined.
        \item Verify that $\mathbf{F}$ is conservative on $D$.
        \item Find a potential function $f$ on $D$
    \end{itemize}
    \textbf{Solution}
    \begin{itemize}
        \item The vector field is undefined for the line $y=0$.
        A simply connected region must not contain any holes and must be in one complete piece;
        therefore $D$ must be all points $(x,y)$ such that $y > 0$ (above the x axis).
        \[\mathcal{D} =  \{(x,y)|\hspace{0.1in} y > 0\}\]
        \item Testing if $\mathbf{F}$ is conservative (for vector fields in $\mathbb{R}^2$ this can be thought of as finding
        the 'curl' and verifying it is 0, although not really).
        \begin{align*}
            P_y &= -\frac{2x}{y^2}\\
            Q_x &= -\frac{2x}{y^2}
        \end{align*}
        $\mathbf{F}$ is conservative on the domain $\mathcal{D}$. To find the potential function $f$, I think integrating $P(x,y)$
        against $x$ is the easier way to start:
        \begin{align*}
            f &= \int P(x,y) dx = \frac{x^2}{y} -e^x + g(y)\\
            f_y &= Q(x,y)\\
            -\frac{x^2}{y^2} +g'(y) &= -\frac{x^2}{y^2}+3y\\
            g'(y) &= 3y \rightarrow g(y) = \frac{3}{2}y^2 + C
        \end{align*}
        The general solution is
        \[f(x,y) = \frac{x^2}{y}+\frac{3}{2}y^2 -e^{x} + C\]
        Any particular solution is a potential function that satisfies $\mathbf{F}(x,y) = \nabla f(x,y)$
    \end{itemize}
\end{document}


